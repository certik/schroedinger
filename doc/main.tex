\documentclass[11pt]{article}
\usepackage[utf8]{inputenc}

\input macros

\title{Nonlinear Formulation of DFT}
\author{Ondřej Čertík, Pavel Šolín, Jiří Vackář}

\begin{document}

\maketitle

\section{Introduction}

This is a solver for the Schr\"odinger equation in the framework of DFT.

\section{Problem}

The task is to find such a charge density $n$, so that all the equations below hold (e.g. are
self-consistent):
$$ V = -{Z\over r} + V_H + V_{xc} $$
$$
\left(-\nabla^2+V\right)\phi_m = \epsilon_m\phi_m,\quad\quad m = 1, 2, \dots, 4
$$
$$ n = \sum_{m=1}^4 \phi_m^2 $$
$$ V_{xc} = f(n) $$
$$ \nabla^2 V_H = n $$

\section{Reformulation}

Let's write everything in terms of $\phi_m(x)$ explicitly:

$$n(x) = %n(y^{(1)}, \dots, y^{(4)}) =
\sum_{m=1}^4 \phi_m^2(x)$$
$$V_{xc}(x) = f(n(x)) = f\left( \sum_{m=1}^4\phi_m^2(x) \right)$$
$$V_H(x) = \half \int_\Omega {n(x')\over|x' - x|}\d x'=
\half \int_\Omega {
\sum_{m=1}^4 \phi_m^2(x')
\over|x' - x|}\d x'
$$
$$V(x) = -{Z\over r} + V_H(x) + V_{xc}(x)=$$
$$
=-{Z\over r}+
\half \int_\Omega {
\sum_{m=1}^4 \phi_m^2(x')
\over|x' - x|}\d x'
+f\left( \sum_{m=1}^4\phi_m^2(x) \right)
$$
Now we can write everything as just one (nonlinear)
equation:
$$
\left(-\nabla^2
-{Z\over r}+
\half \int_\Omega {
\sum_{m=1}^4 \phi_m^2(x')
\over|x' - x|}\d x'
+f\left( \sum_{m=1}^4\phi_m^2(x) \right)
\right)\phi_n = \epsilon_n\phi_n,\quad\quad n = 1, 2, \dots, 4
$$

\section{FE Discretization}

The correspondig discrete problem has the form
$$
\int_\Omega \nabla\phi_n(x)\cdot\nabla v_i(x)+\left[
-{Z\over r}+
\half \int_\Omega {
\sum_{m=1}^4 \phi_m^2(x')
\over|x' - x|}\d x'
+f\left( \sum_{m=1}^4\phi_m^2(x) \right)
\right]
\phi_n(x) v_i(x)  \d x=
$$
$$
=\int_\Omega
\epsilon_n\phi_n(x) v_i(x) \d x,\quad\quad n = 1, 2, \dots, 4;\quad
i = 1, 2, \dots, N
$$
where
$$\phi_n = \phi_n({\bf Y}^{(n)}) = \sum_{j=1}^N y_j^{(n)} v_j(x)$$
Here ${\bf Y}^{(n)} = (y_1^{(n)}, y_2^{(n)}, \dots, y_N^{(n)})^T$ is the vector
of unknown coefficients for the $n$th wavefunction $\phi_n(x)$. Our equation
can then be written in the compact form
$${\bf F}({\bf Y}^{(n)}) = {\bf 0},\quad\quad n = 1, 2, \dots, 4$$
where ${\bf F} = (F_1, F_2, \dots, F_N)^T$ with
$$F_i({\bf Y}^{(n)}) =
\int_\Omega \nabla\phi_n(x)\cdot\nabla v_i(x)+\left[
-{Z\over r}+
\half \int_\Omega {
\sum_{m=1}^4 \phi_m^2(x')
\over|x' - x|}\d x'
+f\left( \sum_{m=1}^4\phi_m^2(x) \right)
\right]
\phi_n(x) v_i(x)  \d x-
$$
$$
-\int_\Omega
\epsilon_n\phi_n(x) v_i(x) \d x
$$

\section{Jacobian}

The Jacobi matrix has the elements:
$$J_{ik} = {\partial F_i\over\partial y_k^{(s)}}$$
The only possible dense term is:
$$
{\partial\over\partial y_k^{(s)}}\int_\Omega \int_\Omega {
\sum_{m=1}^4 \phi_m^2(x')
\over|x' - x|}\d x'\,\phi_n(x) v_i(x) \d x =
$$
$$
=
{\partial\over\partial y_k^{(s)}}\int_\Omega \int_\Omega {
\sum_{m=1}^4 \left(\sum_{j=1}^N y_j^{(m)} v_j(x')\right)^2
\over|x' - x|}\d x'\, \left(\sum_{j=1}^N y_j^{(n)} v_j(x)\right)  v_i(x) \d x =
$$
$$
=
\int_\Omega \int_\Omega {
2 \left(\sum_{j=1}^N y_j^{(s)} v_j(x')\right)v_k(x')
\over|x' - x|}\d x'\, \left(\sum_{j=1}^N y_j^{(n)} v_j(x)\right)  v_i(x) \d x +
$$
$$+
\int_\Omega \int_\Omega {
\sum_{m=1}^4 \left(\sum_{j=1}^N y_j^{(m)} v_j(x')\right)^2
\over|x' - x|}\d x'\, \delta_{ns}v_k(x)  v_i(x) \d x
$$
Now we can see that we have in there the following term:
$$
\int_\Omega \int_\Omega {v_k(x') v_i(x)\over |x'-x|}\d x'\d x
$$
which is dense in $(ki)$, as can be easily seen be fixing $i$ and writing
$$
\int_\Omega \int_\Omega {v_k(x')\over |x'-x|}\d x' v_i(x)\d x
$$
so for each $k$ there is some contribution from the integral $ \int_\Omega
{v_k(x')\over |x'-x|}\d x' $ for such $x$ where $v_i(x)$ is nonzero, thus
making the Jacobian $J_{ik}$ dense.


\end{document}

